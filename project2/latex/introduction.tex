\section{Introduction}\label{sec:introduction}


Supervised learning problems dealing with regression and classification both benefit, or even rely on, optimisation methods for locating minima of some loss function. A commonly used solution in machine learning is the stochastic gradient descent algorithm with its many subvariants. 


\fillertext


Section \ref{sec:theory} presents the theoretical background of the following analysis in section \ref{sec:analysis}. Above lies the nomenclature of this paper, for reference. We summarise our results section \ref{sec:conclusion}. In more detail, in section \ref{sec:stochastic_gradient_descent} we describe the main ideas behind the methods of steepest descent. We move on to describe some basic theory concerning neural networks and the structure of a feedforward neural network in section \ref{sec:neural_network}. We connect this to linear regression in section \ref{sec:regression}, classification in \ref{sec:classification} and logistic regression in \ref{sec:logistic_regression}. In section \ref{sec:validation} we briefly state how models like these are validated in this paper. The analysis part is initiated with a simple regression problem in \ref{sec:analysis_SGD} concerning the steepest descent methods. We present how we build our neural network in \ref{sec:analysis_NN}, which we use to solve the more complex regression problem in section \ref{sec:analysis_regression}. We move on to a binary classification problem which we analyse with our neural network in section \ref{sec:analysis_classification} and with logistic regression in section \ref{sec:analysis_logistic_regression}. We present some closing thoughts \rephrase{(cunundrums? no...)} in section \ref{sec:analysis_closing_words} before we summarise our main results in section \ref{sec:conclusion}.

There are three appendices to this report, \ref{app:regression}, \ref{app:classification} and \ref{app:logistic}, all of which support our analysis through a number of figures. Additional results can be found in the \figureslink, in the figure folder of our Github repository, for the interested reader.