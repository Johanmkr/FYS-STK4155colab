\documentclass{aa} 

\usepackage[utf8]{inputenc}
\usepackage[english]{babel}

\usepackage{physics,amssymb}  
\usepackage{graphicx} 
\usepackage{float}
\usepackage{xcolor}
\usepackage{hyperref} 
\usepackage{array}
%\usepackage{tikz} 
\usepackage{listings}   
\usepackage{url}
\usepackage{upgreek}
\usepackage[super]{nth}
\usepackage{enumitem}
\usepackage{color,soul}
\usepackage{stackengine}
\usepackage{textgreek}
\usepackage{bookmark}
\usepackage{mathrsfs}
\usepackage{nicefrac}
\usepackage[caption=false]{subfig}
\usepackage{bbold}

\graphicspath{{../output/figures}}


\makeatletter
\newcommand\footnoteref[1]{\protected@xdef\@thefnmark{\ref{#1}}\@footnotemark}
\makeatother

\makeatletter
\renewcommand*\aa@pageof{, page \thepage{} of \pageref*{LastPage}}
\makeatother

%\newcommand{\Table}[1]{Table \ref{table:#1}}
%\newcommand{\Fig}[1]{Figure \ref{fig:#1}}
%\newcommand{\Eq}[1]{eq. (\ref{eq:#1})}
%\newcommand{\Sec}[1]{section \ref{sec:#1}}


\definecolor{codegray}{gray}{0.9}
\newcommand{\code}[1]{\colorbox{codegray}{\texttt{#1}}}


\newcommand{\unitC}{ $^\circ$C}
\newcommand{\degs}{$^\circ$}
\newcommand{\pwr}[1]{$^{#1}$} %power
\newcommand{\idx}[1]{$_{#1}$} %index
\newcommand{\stdf}[1]{$\cross 10^{#1}$} %standard form
\newcommand{\unit}[1]{\text{ {#1}}}
\newcommand{\about}{$\sim$}
\newcommand{\ds}{\text{d}}

\renewcommand{\vec}{\mathbf}
\newcommand{\svec}{\boldsymbol}
\renewcommand{\phi}{\varphi}
\renewcommand{\epsilon}{\varepsilon}
\renewcommand{\kappa}{\varkappa}
\renewcommand{\rho}{\varrho}

\newcommand{\F}{\mathscr{F}} %Fourier
\newcommand{\Sun}{_\odot} %Sun mark
\newcommand{\Sk}{\mathcal{S}} %S-function

\newcommand{\hatx}{ \vec{\hat{e}_x} } %unit vec x
\newcommand{\haty}{ \vec{\hat{e}_y} } %unit vec y
\newcommand{\hatz}{ \vec{\hat{e}_z} } %unit vec z

\newcommand{\lap}{\nabla^2 } %Laplacian
\newcommand{\dal}{\Box } %d'Alembertian

\newcommand{\bnabla}{\boldsymbol{\nabla}}


\newcommand{\TT}{^\intercal} % transpose
\newcommand{\RR}[1][]{\mathbb{R}^{#1}} %real numbers
\newcommand{\II}[1][]{\mathbb{1}_{#1}} %identity matrix





\hypersetup{
    colorlinks,
    linkcolor={red!50!black},
    citecolor={blue!50!black},
    urlcolor={blue!80!black}}
    
\lstset{ %
	inputpath=,
	backgroundcolor=\color{white!88!black},
	basicstyle={\ttfamily\scriptsize},
	commentstyle=\color{magenta},
	language=Python,
	morekeywords={True,False},
	tabsize=4,
	stringstyle=\color{green!55!black},
	frame=single,
	keywordstyle=\color{blue},
	showstringspaces=false,
	columns=fullflexible,
	keepspaces=true}   

\begin{document}



\title{Classification and regression: \\
From linear and logistic regression to neural network} 

\author{Nanna Bryne
\inst{1,2}
\and
Johan Mylius Kroken
\inst{1,2}
}
\institute{Institute of Theoretical Astrophysics (ITA), University of Oslo, Norway
\and
Center for Computing in Science Education (CCSE), University of Oslo, Norway}
%\email{nanna.bryne@fys.uio.no}}
\titlerunning{From linear and logistic regression to neural network}
\authorrunning{Bryne\and Kroken} 
\date{\today    \quad GitHub repo link: \url{\projectTwolink}}  
\abstract{
We build a versatile neural network code in order to perform linear regression and binary classification tasks. We train the network by minising the loss function by performing plain and stochastic gradient descent (SGD) for a variety of optimisation algorithms. SGD with RMSProp optimiser perform best and is used in training. A network with 1 hidden layer of 30 neurons where $\eta=10^{-1}$ and $\lambda=10^{-4}$ which uses the sigmoid activation function trained for 700 epochs with 2 minibatches yield the best test MSE of 0.052 when trained to fit the noise Franke function, compared to an MSE of 0.15 for OLS. For the binary classification task the data is the Wisconsin Breast Cancer data. A neural network of 2 hidden layers of 10 neurons each where $\eta=10^{-3}$ and $\lambda=10^{-6}$ which uses the ReLU activation function trained for 900 epochs with 5 minibatches yield the best accuracy of 1. Logistic regression with $\eta=10^{-3}$ and $\lambda=10^{-8}$ also yield an accuracy of 1. 
}

\maketitle


\bibliographystyle{../../aa}

% \par $\quad$ 
% \par \noindent ************************************************

% \feltcute{for ideas (\textbackslash feltcute)}

% \rephrase{rephrase this (\textbackslash rephrase\{...\})}

% \checkthis{check if this is correct (\textbackslash checkthis\{...\})}

% \comment{comment (\textbackslash comment\{...\})}

% \fillertext[(\textbackslash fillertext)]

% \wtf[for when you are lost (\textbackslash wtf)]

% \par \noindent ************************************************
% \par $\quad$ 
% \par $\quad$ 

\tableofcontents
\section*{Notation and nomenclature}

\subsection*{Datasets and fitting} % working title
\begin{itemize}
    
    \item[$\mathcal{D}$] Dataset $\big\{ X, \vec{y} \big\}$ of length $n\in \mathbb{N}$ on the form $\big\{(\vec{x}^{(1)}, y^{(1)}),\,(\vec{x}^{(2)}, y^{(2)}),\,\dots, \, (\vec{x}^{(n)}, y^{(n)}) \big\} $
    \item[$\svec{\theta}$] Parameter vector or vector of coefficients 
    \item[$\mathcal{L}$] Total loss function $\mathcal{L}(\hat{y}, y)$ where $\hat{y}= f(\svec{\theta}; \vec{x})$ (often written as $\mathcal{L}(\svec{\theta})$ for ease of notation)
    \item[$n$] Number of samples in a dataset
    \item[$p$] Number of features of the dependent variables in a dataset
    \item[$X$] Feature matrix of $n$ row vectors $\vec{x} \in \RR[p]$, where $p$ denotes the number of features we are considering
    \item[$\vec{y}$] Vector of $n$ input targets $y^{(i)} \in \RR$ associated with $\vec{x}^{(i)}$
\end{itemize}
\subsection*{Steepest ascent variables}
\begin{itemize}
    \item[$\vec{v}$] Momentum in parameter space, 
    \item[$\mathcal{A}$] Magnitude and direction of steepest ascent in parameter space %\item[$\mathcal{B}$] Subset of $\mathcal{D}$ 
\end{itemize}

\subsection*{Network components}
\begin{itemize}
    \item[$\vec{h}$] Hidden layer of a neural network ($\vec{h}^l \in \RR[N_l]$)
    \item[$g$] Activation function associated with a layer in a neural network, affine transformation ($g_l \,:\,\RR[N_l] \to\RR[N_{l}]$)
    \item[$W$] Matrix of weights describing the mapping from a layer to the next ($W^{l\to l+\! 1} \in \RR[N_l \cross N_{l+\! 1}]$)
    \item[$\vec{b}$] Bias term ($\vec{b}^l \in \RR[N_l]$)
    \item[$\vec{a}$] \checkthis{Activation argument} ($\vec{a}^l \in \RR[N_l]$)
    \item[$N$] Number of neurons in a layer ($N \in \mathbb{N}$)
\end{itemize}

\subsection*{Hyperparameter syntax}
\begin{itemize}
    \item[$\eta$] Learning rate (global)
    \item[$\gamma$] Momentum factor (constant term)
    \item[$L$] Number of layers in an NN, not counting the input layer
    \item[$\lambda$] Regularisation parameter (penalty parameter in Ridge regression)
    \item[$m$] Number of minibatches
\end{itemize}


\subsection*{Miscellaneous}

\begin{enumerate}[leftmargin=4.1em]
    \item[$\norm{\vec{u}}_q$] \lnorm[q]\, of $\vec{u}$
    \item[$\nabla_{\!\xi} J$] Gradient of $J$ with respect to $\svec{\xi}$
    \item[$\mathcal{N}(\mu, \sigma)$]  Normal distribution with mean $\mu$ and standard deviation $\sigma$
\end{enumerate}

\subsection*{Acronyms}
\begin{enumerate}[leftmargin=3.1em]
    \item[DAG] Directed acyclic graph
    \item[FFNN] Feedforward neural network
    \item[GD] Gradient descent
    \item[MSE] Mean squared error 
    \item[NAG] Nesterov accelerated gradient
    \item[NN] Neural network 
    \item[OLS] Ordinary least squares 
    \item[ReLU] Rectified linear unit
    \item[SGD] Stochastic gradient descent 
\end{enumerate}


\comment{Should order alphabetically or logically.}

%\tableofcontents
\section{Introduction}\label{sec:intro}
Linear regression is the gateway to statistical analysis, and in this investigation we will perform three types of linear regression: ordinary least squares (OLS), Ridge, and Lasso regression, on two different data sets: The Franke function and terrain data of parts of the Grand Canyon in the United States. Common for all is, in addition to minimising error, that we will fit a two-dimensional polynomial onto the two data sets by finding an optimal set of parameters $\optbeta$, which in our case will be the coefficients of the different terms in the polynomial. How we determine these parameters depend on the regression method of choice. The way of measuring error (deviation from data), the cost function, varies between the different regression methods. Ridge and Lasso regression are so-called regularisation methods, allowing us to deal with more complex models, reducing the chance of over-fitting. Determining the best model is very dependant on the data set at hand. In \Sec{SVD} we state the singular value decomposition from linear algebra. This comes in handy when explaining the regression models in \Sec{regression}, especially the penalty term involved in the regularisation of the Ridge and Lasso schemes. We also explain resampling methods in \Sec{resampling}, which are useful when assessing the accuracy of our model. The main analysis is given in \Sec{analysis} with an introduction to the data and noise of the Franke function in \Sec{data}, description of how and why we split the data in \Sec{splitting}, how we set up the model and design matrix in \Sec{model}, and how and why we scale the data in \Sec{scaling}. We then carry out the analysis of the Franke function in \Sec{reganalysis_franke}, and of the terrain data in \Sec{reganalysis_real_data}. We draw conclusions in \Sec{conclusion}. At last, we link to the code and list all figures, as well as captions.
\section{Theory}\label{sec:theroy}




\hl{*} DECIDE ON z vs. y

\begin{align}\label{eq:MSE}
    \MSE{\vec{y}, \tilde{\vec{y}}}= \frac{1}{n}(\vec{y}-\tilde{\vec{y}})^2= \frac{1}{n}\sum_{i=0}^{n-1} (y_i - \tilde{y}_i)^2
\end{align}

\begin{align}\label{eq:R2}
    \Rtwo {\vec{y}, \tilde{\vec{y}}} =1 - \frac{(\vec{y}-\tilde{\vec{y}})^2}{(\vec{y}(1-\bar{y}))^2}= 1 - \frac{\sum_{i=0}^{n-1}(y_i-\tilde{y}_i)^2}{\sum_{i=0}^{n-1}(y_i-\bar{y})^2}
\end{align}

\begin{align*}
    \bar{y} = \frac{1}{n}\sum_{i=0}^{n-1} y_i
\end{align*}



\subsection{Linear regression}\label{sec:regression}

There are several possible estimation techniques when fitting a linear regression model. We will discuss three common approaches, one least squares estimation (\Sec{OLS}) and two forms of penalized estimation (\Sec{Ridge} and \Sec{Lasso}).

We assume the vector $\vec{y} \in \RR[n]$ consisting of $n$ observed values $y_i$ to take the form $\vec{y}=f(\vec{x})+\svec{\epsilon}$ where $f(\vec{x})\in\RR[n]$ is a continous function and $\svec{\epsilon}\in \RR[n] $ is a normally distributed noise of standard deviation $\sigma$. We approximate $f$ by $\tilde{\vec{y}}=X\svec{\beta}$, where $X\in \RR[n\cross p]$ is the design matrix of $n$ row vectors $\vec{x}_i\in \RR[p]$, and $\svec{\beta}\in \RR[p]$ are the unknown parameters to be determined. That is, we assume a \textit{linear} relationship between $X$ and$\vec{y}$. The integers $n$ and $p$ then represent the number of data points and features, respectively. 

For an observed value $y_i$ we have $y_i = \vec{x}_i\TT \svec{\beta} + \epsilon_i = \Xbi+ \epsilon_i$. The inner product $\Xbi$ is non-stochastic, hence 

\begin{align*}
    \EE{\Xbi} = \Xbi
\end{align*}

and since 

\begin{align*}
    \EE{\epsilon_i} \stackrel{\text{per def.}}{=} 0,
\end{align*}

we have the expectation value of the response variable

\begin{align*}
    \EE{y_i} &= \EE{\Xbi+ \epsilon_i} \\
    &= \EE{\Xbi} + \EE{\epsilon_i} \\
    &= \Xbi.
\end{align*}

To find the variance of this dependent variable, we need the expetation value of the outer product $\vec{y}\vec{y}\TT$,

\begin{align}
    \EE{\vec{y} \vec{y}\TT} &= \EE{(X\svec{\beta} + \svec{\epsilon})(X\svec{\beta} + \svec{\epsilon})\TT} \nonumber\\
    &= \EE{\Xb \svec{\beta}\TT X\TT + \Xb \svec{\epsilon}\TT + \svec{\epsilon}\svec{\beta}\TT X\TT + \svec{\epsilon} \svec{\epsilon}\TT} \nonumber \\
    &= \Xb \svec{\beta}\TT X\TT + \II \sigma^2. \label{eq:expectation_yyT}
\end{align}

The variance becomes

\begin{align*}
    \variance{y_i} &= \EE{(\vec{y}\vec{y}\TT)_{ii}} -\Big(\EE{y_i}\Big)^2\\
    &= \Xbi \Xbi + \sigma^2 - \Xbi \Xbi\\
    &= \sigma^2.
\end{align*}


The optimal estimator of the coefficients $\svec{\beta}_j$, call it $\optbeta$, is in principle obtained by minimizing the cost function $C(\svec{\beta})$, and said function is determined by the method we choose. That is,

\begin{align}\label{eq:general_LS}
    \pdv{C(\svec{\beta})}{\svec{\beta}}\Bigg|_{\svec{\beta}=\optbeta} = 0.
\end{align}

\subsubsection{Ordinary Least Squares (OLS)}\label{sec:OLS}

The ordinary least squares (OLS) method assumes the cost function

\begin{align*}
    C^\text{OLS}(\svec{\beta}) = \norm{\vec{y}-\tilde{\vec{y}}}_2^{2} = \norm{\vec{y}-\Xb}_2^{2},
\end{align*}

where the subscript "2" implies the \lnorm{2}. Solving \Eq{general_LS} for $C=C^\text{OLS}$ yields the OLS expression for the optimal parameter

\begin{align*}
    \optbeta^\text{OLS} = \invhessian X\TT \vec{y} = H^{-1} X\TT \vec{y},
\end{align*}

where $H = \hessian$ is the Hessian matrix.

Letting $\optbeta = \optbeta^\text{OLS}$ we get the expected value 

\begin{align*}
    \EE{\optbeta} &= \EE{\invhessian X\TT \vec{y}} \\
    &= \invhessian X\TT \EE{\vec{y}} \\
    &= \invhessian \hessian \svec{\beta} \\
    &= \svec{\beta}.
\end{align*}

The variance is then 

\begin{align*}
    \variance{\optbeta} &= \EE{\optbeta \optbeta\TT} -\EE{\optbeta} \EE{\optbeta\TT} \\
    &= \EE{\invhessian X\TT  \vec{y} \vec{y}\TT X (\invhessian)\TT} - \svec{\beta} \svec{\beta}\TT \\
    &= \invhessian X\TT \EE{ \vec{y} \vec{y}\TT } X \invhessian - \svec{\beta} \svec{\beta}\TT \\
    &\stackrel{\text{\eqref{eq:expectation_yyT}}}{=}\invhessian X\TT ( X\svec{\beta} \svec{\beta}\TT X\TT+ \II\sigma^2) X \invhessian \\
    &= \svec{\beta} \svec{\beta} \TT + \invhessian X\TT \sigma^2 X \invhessian - \svec{\beta} \svec{\beta} \TT \\
    &= \sigma^2 \invhessian.
\end{align*}



\dots



\subsubsection{Ridge regression}\label{sec:Ridge}

Let $\lambda \in \RR$ be some small number such that $\lambda >0$. If we add a penalty term $\lambda \norm{\svec{\beta}}_2^2$ to the OLS cost function, we get the cost function of Ridge regression,

\begin{align*}
    C^\text{Ridge}(\svec{\beta}) &=  C^\text{OLS}(\svec{\beta}) + \lambda \norm{\svec{\beta}}_2^2 \\
    &=\norm{\vec{y}-\tilde{\vec{y}}}_2^{2}  + \lambda \norm{\svec{\beta}}_2^2 \\
    &= \norm{\vec{y}-\Xb}_2^{2} + \lambda \norm{\svec{\beta}}_2^2 
\end{align*}

\begin{align*}
    \optbeta^\text{Ridge} = \big(\hessian + \lambda \II\big)^{-1} \vec{y} = \big(H + \lambda \II\big)^{-1} \vec{y}
\end{align*}


\subsubsection{Lasso regression}\label{sec:Lasso}

If we add the penalty term $\lambda \norm{\svec{\beta}}_1$, now using the \lnorm{1}, to the OLS cost function, we are left with the Lasso regression's cost function,
\begin{align*}
    C^\text{Lasso}(\svec{\beta})  &= C^\text{OLS}(\svec{\beta}) + \lambda \norm{\svec{\beta}}_1 \\
    &= \norm{\vec{y}-\tilde{\vec{y}}}_2^{2}  + \lambda \norm{\svec{\beta}}_{1} \\
    &= \norm{\vec{y}-\Xb}_2^{2} + \lambda \norm{\svec{\beta}}_{1}
\end{align*}

\begin{align*}
    \optbeta^\text{Lasso} = \dots
\end{align*}

\subsection{Resampling}\label{sec:resampling}

\subsubsection{Bootstrap method}\label{sec:bootstrap}

\subsubsection{Cross-validation}\label{sec:k_fold}

\section{Analysis}\label{sec:analysis}

\comment{Present data sets}

$\datasetA = \big\{ (\vec{x}^{(1)}, y^{(1)}), \,  (\vec{x}^{(2)}, y^{(2)}), \,\dots, \, (\vec{x}^{({\npointsA})}, y^{(\npointsA)}) \big\}$ 

\comment{Maybe write something about the codes?}

\subsection{Regression problem}\label{sec:analysis_regression}

    \subsubsection{Pre-NN}\label{sec:analysis_regressoin_preNN}
    \comment{Fix title}

    Using the \checkthis{GD/SGD} method, we perform an OLS regression on the data set $\datasetA$ by using the cost function in \ref{eq:ols_cost_function}.




\subsection{Classification problem}\label{sec:analysis_classification}


\section{Conclusion}\label{sec:conclusion}


We present our final models for the Franke function and the Grand Canyon terrain. 


\subsection{Franke function}
 
After performing the OLS analysis on the Franke function, it quickly became clear that a two-dimensional polynomial of order $d=5$ was the best model, i.e. yielding the lowest MSE. This can be seen most clearly from \Fig{model_complexity_ols} and \Fig{bias_variance_ols} we compare model of both higher and lower polynomial degree. A rough by-eye estimate of the MSE in these two cases is $\mathrm{MSE}\approx 0.16$ for the test data. This is confirmed in \Fig{cross-validation_ols} and \Fig{mse_hist_ols}, where the former yield a slightly higher MSE value. 

For the ridge analysis we just assume that the polynomial degree found for OLS will yield the best result, and seek to find the optimal $\lambda$ only. One could argue that this is naive, so we should perhaps have allowed for the polynomial degree to change when performing ridge analysis. However, with $d=5$, the optimal penalty parameter is found to be $\lambda_\mathrm{min}^\mathrm{ridge} = 7.85\cdot10^{-5}$ given to two significant figures. We have from \Fig{bootstrapping_ridge} and \Fig{mse_hist_ridge} that an approximate value of the MSE found for this model is $\mathrm{MSE} \approx 0.14$ for the test data. 

for the lasso analysis we try to find a new optimal polynomial degree, which result in the maximum value of our grid $d=14$, which is not necessarily a bad model for a $N=20$ grid, since the penalty parameter drives unimportant features to zero. However, according to \Fig{cross-validation_lasso} and \Fig{bias_variance_lasso} there seem to be an optimal $\lambda\in[10^{-5}, 10^{-4}]$. However, the MSE value of $\approx 0.15$ of the same order for both OLS and ridge. 

One could argue that ridge yield a slightly lower MSE than OLS. However, these values are highly dependant on the method by which they were found, and the $k$-number in both bootstrap and cross validation. The only sensible conclusion is thus that they yield approximately the same result. While OLS is less computationally expensive, and gives a simpler model altogether we further conclude that OLS with a polynomial degreee $d=5$ is the model that fits the data best. 

We check this and plot the prediction model alongside the scaled data points. This can be seen in \Fig{franke_final_model} and the model is indeed good. 

\begin{figure}
    \includegraphics[width=\linewidth]{Franke/comparison3D_ols.pdf}
    \caption{Best fit model of the Franke function. The triangular points are the scaled data points, whilst the surface represent the corresponding prediction we get from an OLS estimate with $d=5$. }
    \label{fig:franke_final_model}
\end{figure}



\subsection{Grand Canyon terrain}

We got two models that in terms of MSE were almost equally good. The OLS model has $d^\text{OLS}=6$ and the Ridge model has $d^\text{Ridge}=18$ and $\lambda^\text{Ridge}=1.23\cdot10^{-4}$. The analysis yielded very similar prediciton errors for the two, see \Fig{gc_model_complexity_ols} and \Fig{gc_model_complexity_ridge}. However, as introduced in \Sec{Ridge}, the variance of the $\beta_j$'s are generally lower for the Ridge scheme than for the OLS scheme, and this is why we choose the former. The resulting terrain prediction is shown in \Fig{gc_final_model}.


\begin{figure}
    \includegraphics[width=\linewidth]{terrain/comparison3D_ridge.pdf}
    \caption{Best fit model of the terrain data. The triangular points are the scaled test data points, whilst the surface represents the corresponding prediction we get from a Ridge estimation with $d^\text{Ridge}=18$ and $\lambda^\text{Ridge}=1.23\cdot 10^{-4}$.}
    \label{fig:gc_final_model}
\end{figure}

The terrain is definitely recognisable to the Grand Canyon (\Fig{gc_data}, mind the angle). There is a tendency to flatten out by the edges as opposed to continuing downwards, which was the typical case with the OLS-models.


\section*{Code availability}
The code is available on GitHub at \url{\projectTwolink}.

%\newpage
%\listoffigures

\bibliography{ref}

\clearpage



\appendix

\section{Regression figures}\label{app:regression}

\begin{figure}[h!]
    \includegraphics[width=\linewidth]{eta_lambda_analysis.pdf}
    \caption{Heatmap of the MSE as function of learning rate $\eta$ and regularisation parameter $\lambda$, using SGD with RMSProp as optimiser performing regression analysis of a 3 layered, 15-10-5 neurons, neural network. }
    \label{fig:reg_eta_lambda}
\end{figure}

\begin{figure}[h!]
    \includegraphics[width=\linewidth]{layer_neuron_analysis.pdf}
    \caption{Heatmap of the MSE as function of hidden layers $L-1$ and neurons per layer $N_l$, using SGD with RMSProp as optimiser performing regression analysis with $\eta=10^{-1}$ and $\lambda=10^{-4}$.}
    \label{fig:reg_layer_neuron}
\end{figure}

\begin{figure}[h!]
    \includegraphics[width=\linewidth]{actFuncPer1000Epoch.pdf}
    \caption{Plot of the MSE for up to 1000 epochs, using SGD with RMSProp as optimiser performing regression analysis with $L-1=1$ hidden layer with $N_l=30$ neurons with $\eta=10^{-1}$ and $\lambda=10^{-4}$. The four different activation functions perform differently. Note the logarithmic MSE axis.}
    \label{fig:reg_act_epoch1000}
\end{figure}

\begin{figure}[h!]
    \includegraphics[width=\linewidth]{actFuncPerEpoch.pdf}
    \caption{Plot of the MSE for up to 250 epochs, using SGD with RMSProp as optimiser performing regression analysis with $L-1=1$ hidden layer with $N_l=30$ neurons with $\eta=10^{-1}$ and $\lambda=10^{-4}$. The four different activation functions perform differently. Note the logarithmic MSE axis.}
    \label{fig:reg_act_epoch}
\end{figure}

\begin{figure}[h!]
    \includegraphics[width=\linewidth]{epoch_minibatch_analysis.pdf}
    \caption{Heatmap of the MSE as function of the number of minibatches $m$ and training epochs, using SGD with RMSProp as optimiser performing regression analysis with $L-1=1$ hidden layer with $N_l=30$ neurons with $\eta=10^{-1}$ and $\lambda=10^{-4}$ using sigmoid as activation function. }
    \label{fig:reg_minibatch_epoch}
\end{figure}




\clearpage

\section{Classification figures}\label{app:classification}

\begin{figure}[h!]
    \includegraphics[width=\linewidth]{eta_lambda_analysisCancer.pdf}
    \caption{Heatmap of accuracy as function of learning rate $\eta$ and regularisation parameter $\lambda$, using SGD with RMSProp as optimiser performing regression analysis of a 3 layered, 15-10-5 neurons, neural network. }
    \label{fig:class_eta_lambda}
\end{figure}

\begin{figure}[h!]
    \includegraphics[width=\linewidth]{layer_neuron_analysisCancer.pdf}
    \caption{Heatmap of accuracy as function of hidden layers $L-1$ and neurons per layer $N_l$, using SGD with RMSProp as optimiser performing regression analysis with $\eta=10^{-3}$ and $\lambda=10^{-6}$.}
    \label{fig:class_layer_neuron}
\end{figure}

\begin{figure}[h!]
    \includegraphics[width=\linewidth]{actFuncPerEpoch1000Cancer.pdf}
    \caption{Plot of accuracy for up to 1000 epochs, using SGD with RMSProp as optimiser performing regression analysis with $L-1=2$ hidden layers with $N_l=10$ neurons each with $\eta=10^{-3}$ and $\lambda=10^{-6}$. The four different activation functions perform differently.}
    \label{fig:class_act_epoch1000}
\end{figure}

\begin{figure}[h!]
    \includegraphics[width=\linewidth]{actFuncPerEpochCancer.pdf}
    \caption{Plot of accuracy for up to 250 epochs, using SGD with RMSProp as optimiser performing regression analysis with $L-1=2$ hidden layer with $N_l=10$ neurons with $\eta=10^{-3}$ and $\lambda=10^{-6}$. The four different activation functions perform differently.}
    \label{fig:class_act_epoch}
\end{figure}

\begin{figure}[h!]
    \includegraphics[width=\linewidth]{epoch_minibatch_analysisCancer.pdf}
    \caption{Heatmap of accuracy as function of the number of minibatches $m$ and training epochs, using SGD with RMSProp as optimiser performing regression analysis with $L-1=2$ hidden layer with $N_l=10$ neurons with $\eta=10^{-3}$ and $\lambda=10^{-6}$ using RELU as activation function. }
    \label{fig:class_minibatch_epoch}
\end{figure}


\clearpage

\section{Logistic regression figure}\label{app:logistic}

\begin{figure}[h!]
    \includegraphics[width=\linewidth]{logistic.pdf}
    \caption{Heatmap of the MSE as function of learning rate $\eta$ and regularisation parameter $\lambda$, using SGD with RMSProp as optimiser performing regression analysis of a network with no hidden layers using the sigmoid function as activation function. This is equivalent of performing logistic regression.}
    \label{fig:logistic_eta_lambda}
\end{figure}


\clearpage

\section{Optimiser algorithms}\label{app:optimisers}


We have the parameter update 
\begin{equation}\label{eq:update_rule_opt}
    \svec{\theta}_{k+1} = \svec{\theta}_k + \vec{v}_k, \quad k=0,1,\dots, (m\times\#\mathrm{epochs})-1
\end{equation}
for which $\vec{v}_k$ can be computed in different ways. We present three update rules of the adaptive learning rate sort. All require:
\begin{enumerate}[label=*]
    \item a global learning rate $\eta$
    \item a small number $\epsilon$ for numerical stability
    \item an initial parameter $\svec{\theta}_0$
    \item an initial accumulation variable $\vec{r}_{-1}$ (and $\vec{s}_{-1}$ for Adam)
\end{enumerate}


\begin{enumerate}[leftmargin=0pt,labelwidth=!,labelsep=.05em]
    \item[] The \textbf{AdaGrad} algorithm has the following update rule, in combination with eq. \eqref{eq:update_rule_opt}:
    \begin{subequations}\label{eq:adagrad_algo}
    \begin{align}
        \vec{r}_{k} &= \vec{r}_{k-1} + \mathcal{A}_k \odot \mathcal{A}_k \,;\\
        \vec{v}_{k} &= -\frac{\eta}{\epsilon + \sqrt{\vec{r}_k}} \odot \mathcal{A}_k \,;\label{eq:v_ada}
    \end{align}
    \end{subequations}
    \item[] The \textbf{RMSProp} algorithm needs a decay rate $\rho$, and in combination with eq. \eqref{eq:update_rule_opt} updates the parameter as follows:
    \begin{subequations}\label{eq:rmsprop_algo}
    \begin{align}
        \vec{r}_{k} &= \rho \vec{r}_{k-1} + (1-\rho)\mathcal{A}_k \odot \mathcal{A}_k\,; \\
        \vec{v}_{k} &= -\frac{\eta}{\sqrt{\epsilon + \vec{r}_k}} \odot \mathcal{A}_k\,;\label{eq:v_rms}
    \end{align}
    \end{subequations}
    \item[] The \textbf{Adam} algorithm requires two decay rates $\rho_1, \rho_2 \in [0, 1)$ and uses the following scheme to find the parameter change in \eqref{eq:update_rule_opt}:
    \begin{subequations}\label{eq:adam_algo}
    \begin{align}
        \vec{s}_{k} &= \rho_1 \vec{s}_{k-1} + (1-\rho_1)\mathcal{A}_k\,; \\
        \vec{r}_{k} &= \rho_2 \vec{r}_{k-1} + (1-\rho_2)\mathcal{A}_k \odot \mathcal{A}_k\,; \\
        \hat{\vec{s}} &= \frac{\vec{s}_k}{1-\rho_1^{k+1}}\,; \\
        \hat{\vec{r}} &= \frac{\vec{r}_k}{1-\rho_2^{k+1}}\,; \\
        \vec{v}_{k} &= -\frac{\eta \hat{\vec{s}}}{\epsilon+\sqrt{\hat{\vec{r}}}} \,; \label{eq:v_adam}
    \end{align}
    \end{subequations}
\end{enumerate}


In eqs. \eqref{eq:v_ada} the division and square root operations are applied element-wise. 




\section{Back propagation algorithm}\label{app:backprop}
    The output $\vec{\hat{y}}$ of our NN is given by the layer value of the output layer $\vec{h}^L$ passed through an output function $g_L$, s.t. $\vec{\hat{y}} = g_L(\vec{h}^L)$. The output error is given by
    \begin{equation}\label{eq:app_backprop_output_error}
        \svec{\delta}^L = \dv{g_L(\vec{a}^L)}{\vec{a}^L} \odot \pdv{\mathcal{L}}{\vec{\hat{y}}} =\dv{g_L(\vec{a}^L)}{\vec{a}^L} \odot \pdv{\mathcal{L}}{\vec{h}^L}\left( \pdv{\vec{\hat{y}}}{\vec{h}^L} \right)^{-1},
    \end{equation}
    where $\odot$ is the element-wise Hadamard product. This error is propagated through the layers of the network in backward order,
    \begin{equation}\label{eq:app_backprop_prop_error}
        \svec{\delta}^l = \svec{\delta}^{l+1}W^{l+\!1\leftarrow l} \odot \dv{}{\vec{a}^l}g_l(\vec{a}^l),
    \end{equation}
    for $l=L-1, L-2,\dots, 1$. Having found the error propagated through each layer, we update the weights and biases as follows:
    \begin{equation}\label{eq:app_backprop_update}
        \begin{split}
            \nabla_{\!W^{l-\!1\to l}} \mathcal{L} &= \svec{\delta}^l\big[\vec{h}^{l-1}\big]\TT\,; \\
            \nabla_{\!\vec{b}^l} \mathcal{L}&= \svec{\delta}^l\,; \\
            W^{l-\!1\to l} &= \mathcal{U}(W^{l-\! 1 \to l}, \nabla_{\!W^{l-\! 1\to l}} \mathcal{L})\,;\\
            \vec{b}^l &= \mathcal{U}(\vec{b}^l, \nabla_{\!\vec{b}^l} \mathcal{L})\,; \\
            \implies \Theta^l &= \mathcal{U}(\Theta^l, \nabla_{\!\Theta^l}\mathcal{L}) \,;\quad\quad \Theta^l \equiv (W^{l-\! 1 \to l}, \vec{b}^l) \,;
        \end{split}
    \end{equation}
    Here $\mathcal{U}(\Theta, \nabla_{\!\Theta}\mathcal{L})$ is a function that updates the parameter $\Theta$ according som some optimisation scheme, typically SGD with a favoured optimised. For reference, plain gradient descent reads $\mathcal{U}(\Theta, \nabla_{\!\Theta}\mathcal{L}) =\Theta - \eta\nabla_{\!\Theta}\mathcal{L}$, where $\eta$ is the learning rate. 




\end{document}
